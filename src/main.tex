\documentclass{article}
\usepackage[margin=2.54cm]{geometry}
\usepackage{listings}
\usepackage{xcolor}

\definecolor{codegray}{rgb}{0.5,0.5,0.5}
\definecolor{codegreen}{rgb}{0,0.6,0}
\definecolor{backcolour}{rgb}{0.95,0.95,0.92}
\lstdefinestyle{vscStyle}{
    backgroundcolor=\color{backcolour},
    commentstyle=\color{codegreen},
    keywordstyle=\color{blue},
    numberstyle=\tiny\color{codegray},
    stringstyle=\color{orange},
    basicstyle=\ttfamily\footnotesize,
    breakatwhitespace=false,
    breaklines=true,
    captionpos=b,
    keepspaces=true,
    numbers=left,
    numbersep=5pt,
    showspaces=false,
    showstringspaces=false,
    showtabs=false,
    tabsize=2
}
\lstset{style=vscStyle}


\title{QuadraticEquation.java Documentation}
\author{FRC Team 3161 Tronic Titans}

\begin{document}

\maketitle


\begin{lstlisting}[language=Java]

public class QuadraticEquation {
\end{lstlisting}

 coefficients of quadratic equation in standard form $ax^2 + bx + c = 0$

\begin{lstlisting}[language=Java]
    private final double a,b,c;

\end{lstlisting}

 constructor method to set the coefficients of the quadratic in standard form

\begin{lstlisting}[language=Java]
    public QuadraticEquation(double a, double b, double c){
        this.a = a;
        this.b = b;
        this.c = c;
    }

\end{lstlisting}

 retrieval method for coefficient a

\begin{lstlisting}[language=Java]
    public double getA(){
        return a;
    }

\end{lstlisting}

 retrieval method for coefficient b

\begin{lstlisting}[language=Java]
    public double getB(){
        return b;
    }

\end{lstlisting}

 retrieval method for coefficient c

\begin{lstlisting}[language=Java]
    public double getC(){
        return c;
    }

\end{lstlisting}

 method to return zeroes of quadratic equation making use of the quadratic formula
 $x_{1,2} = \frac{-b \pm \sqrt{b^2 - 4ac}}{2a}$

\begin{lstlisting}[language=Java]
    public double[] getRoots(){
        double disc = Math.pow(getB(),2) - 4*getA()*getC(); // the discriminant

        if (disc >= 0){
            double[] roots = new double[2];
            roots[0] = (-1*getB() + Math.sqrt(disc)) / (2*getA()); // the root using +
            roots[1] = (-1*getB() + Math.sqrt(disc)) / (2*getA()); // the root using -
            return roots;
        } else {
            return null; // the case of no roots
        }
    }

\end{lstlisting}

 method to print the x and y coordinates of the vertex of the parabola

\begin{lstlisting}[language=Java]
    public void printVertex(){
        double[] vertex = new double[2];
        vertex[0] = -1 * getB() / (2*getA()); // x coordinate
        vertex[1] = getA()*Math.pow(vertex[0],2) + getB()*vertex[0] + getC(); // y coordinate

        System.out.println("The vertex of the parabola is located at (x,y) = " + Arrays.toString(vertex));
    }
}
\end{lstlisting}

\end{document}